\documentclass[aspectratio=169]{beamer}
\usepackage{cmap}
\usepackage[utf8]{inputenc}
\usepackage[T2A]{fontenc}
\usepackage[russian]{babel}

\usetheme{Berkeley}

\title[Стохасти- ческая задача формирования теста]{Стохастическая постановка задачи формирования теста заданного уровня сложности с минимизацией квантили времени выполнения}
\author[Королев Егор \and Туманов Георгий]{Е.В.Королев \and Г.А.Туманов}
\institute[НИУ МАИ]{Московский авиационный институт (НИУ)}

\begin{document}
    \begin{frame}
        \maketitle
    \end{frame}

    \begin{frame}{План презентации}
        \tableofcontents
    \end{frame}

    \section{Введение}
    \begin{frame}{Введение}
        \begin{block}{LMS}
            С переходом на дистанционное образование активно развиваются системы управления обучением (LMS)\\
        \end{block}
    
        \begin{block}{Способ повышения качества дистанциооного образования}
            Применение анализа данных в LMS может повысить качество дистанционного образования\\
        \end{block}
    
        \begin{block}{За счет чего повышактся качество?}
            Сложность заданий оценивается экспертами, либо программно. Происходит адаптация контента под оцениваемый уровень знаний пользователя\\
        \end{block}
    \end{frame}

    \section{Постановка задачи}
    \begin{frame}{Постановка задачи}
        \begin{block}{Логнормальная модель Ван дер Линдена}
            Пусть $Z=(z_1,\cdots,z_I)$ -- вектор заданий\\
            Ван дер Линден предположил, что логарифм времени $T_j^i$ (время ответа $j$-го пользователя на $i$-ю задачу) состоит из 3-х компонент:
            \begin{itemize}
                \item $\mu$ -- общая составляющая для всех пользователей и задач;
                \item $\beta_i$ -- индивидуальная сложность $i$-й задачи;
                \item $\tau_j$ -- особенности $j$-го пользователя, решающего задание.
            \end{itemize}
            Модель имеет вид:
            \begin{equation}
            \ln T_j^i = \mu + \beta_i + \tau_j + \varepsilon_{ij},
            \end{equation}
            где $\varepsilon_{ij} \thicksim \mathcal{N}(0,\sigma^2)$ -- независимые СВ
        \end{block}
    \end{frame}

    \begin{frame}{Постановка задачи}
        \begin{block}{Оценки параметров модели}
            Из ММП можно получить оценки модели:
            \begin{gather}
                \hat{\mu} = \dfrac{1}{I \cdot J} \sum\limits_{j=1}^J \sum\limits_{i=1}^I \ln t_j^i, ~~~  \hat{\beta_i} = \dfrac{1}{J} \sum\limits_{j=1}^J \ln t_j^i - \hat{\mu}, ~~~  \hat{\tau_j} = \dfrac{1}{I \cdot J} \sum\limits_{i=1}^I \ln t_j^i - \hat{\mu},\\
                \hat{\sigma^2} = \dfrac{1}{I \cdot J} \sum\limits_{j=1}^J \sum\limits_{i=1}^I \left( \ln t_j^i - \hat{\mu} - \hat{\beta_i} - \hat{\tau_j} \right)^2
            \end{gather}
        \end{block}
    \end{frame}

    \begin{frame}{Постановка задачи}
        \begin{block}{Плотность вероятности логнормального распределения}
            Таким образом, из модели (1) и с учетом оценок (2), (3) в качестве модели времени ответа пользователя на задание можно выбрать модель логнормального распределения с плотностью вероятности:
            \begin{equation}
                f(x,\tau_j,\beta_i,\sigma) = \dfrac{1}{x\sigma \sqrt{2\pi}}\exp \left\{ -\dfrac{1}{2} \left[ \dfrac{\ln x - (\hat{\mu} + \hat{\beta_i} + \hat{\tau_j})}{\hat{\sigma}} \right]^2 \right\}
            \end{equation}
        \end{block}
    \end{frame}


    \section{Численный эксперимент}
    \begin{frame}{Численный эксперимент}
        текст\\
    \end{frame}

    \section{Заключение}
    \begin{frame}{Заключение}
        текст\\
    \end{frame}
    
    \begin{frame}{Список литературы}
        текст\\
    \end{frame}

    \begin{frame}
        \centering
        \huge
        Спасибо за внимание!\\
    \end{frame}
\end{document}