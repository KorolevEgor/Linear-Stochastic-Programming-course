\documentclass[aspectratio=169]{beamer}
\usepackage{cmap}
\usepackage[utf8]{inputenc}
\usepackage[T2A]{fontenc}
\usepackage[russian]{babel}

\usetheme{Copenhagen}

\title[стохастическая задача формирования теста]{Стохастическая постановка задачи формирования теста заданного уровня сложности с минимизацией квантили времени выполнения}
\author[Королев Егор \and Туманов Георгий]{Е.В.Королев \and Г.А.Туманов}
\institute[НИУ МАИ]{Московский авиационный институт (НИУ)}

\begin{document}
    \begin{frame}
        \maketitle
    \end{frame}

    \begin{frame}{название слайда}
        текст\\
    \end{frame}

    \begin{frame}{Задача квантильной оптимизации}
        $$u_\alpha = Arg\min_{u\in\{0, 1\}^I}\left(\frac{\gamma\left|c-w^T u\right|}{\varepsilon} + \frac{(1-\gamma)\Phi_\alpha(u)}{2700}\right)$$
        $$\varphi_\alpha = \min_{u\in\{0, 1\}^I}\left(\frac{\gamma\left|c-w^T u\right|}{\varepsilon} + \frac{(1-\gamma)\Phi_\alpha(u)}{2700}\right)$$
        $$c-w^T u\leq\varepsilon$$
        $$w^T u-c\leq\varepsilon$$
        $$A^T u\geq e_M$$
        $$e_I^T u=k$$
    \end{frame}
    
    \begin{frame}{Дискретный аналог}
        Заменим непрерывные случайные величины $T_n^i$ на дискретные $\Theta_n^i$ со следующими распределениями:\\
        
        \begin{table}[]
		\begin{tabular}{|l||l|l|l|l|}
			\hline
			$\Theta_n^i(\lambda)$ & $\theta_n^i(1)$ & $\theta_n^i(2)$ & ... & $\theta_n^i(L_{ni})$\\ \hline
			$p_n^i(\lambda)$ & $p_n^i(1)$ & $p_n^i(2)$ & ... & $p_n^i(L_{ni})$\\ \hline
		\end{tabular}
		\end{table}
		
		где:\newline
		$0 < t_1 < t_2 < ... < t_{L_{ni}-1} < +\infty$ -- разбиение временного интервала\newline
		
		$\theta_n^i(\lambda)$ -- середины интервалов $[t_{\lambda-1}, t_\lambda], l=2,...,L_{ni}-1$\newline
		
		$\theta_n^i(1)$ и $\theta_n^i(L_{ni})$ -- квантили $T_n^i$ уровней 0.01 и 0.99 соответственно\newline
		
		$p_n^i(\lambda)=\int_{t_{\lambda-1}}^{t_\lambda}f(t, \tau_n, \beta_i, \sigma)dt$
    \end{frame}
    
    \begin{frame}{Функция квантили}
        Вместо матрицы $T$ будем использовать мктрицу $\Theta=||\Theta_n^i||$.
        
        Обозначим $\Theta_n$ -- n-ая строка матрицы $\Theta$. Тогда функция квантили примет вид:
        
        $$\Phi_\alpha(u)\triangleq\min\left\{\varphi:P\left\{\max_{n=\overline{1,N}}\Theta_n u\leq\varphi\right\}\geq\alpha\right\}$$
    \end{frame}
    
    \begin{frame}{Сведение к детерминированной задаче}
        Введём следующие обозначения:
        
        $$D=\prod_{n=1}^N\prod_{i=1}^I L_{ni}$$
        
		$\theta_d, d=1,...,D$ -- реализации случайной матрицы $\Theta$
        
        $$p=(p_1,...,p_D), p_d = P(\Theta=\theta_d)=\prod_{n=1}^N\prod_{i=1}^I P(\Theta_n^i=(\theta_d)_n^i)$$
        
        $$\overline{\varphi}=(\varphi,...,\varphi)^T\in R^N$$
        
        $\delta=(\delta_1,...,\delta_D)\in\{0,1\}^D$ -- вектор булевых переменных, определяющий доверительное множество
    \end{frame}

    \begin{frame}{Детерминированная задача}
        $$u^* = Arg\min_{u\in\{0, 1\}^I, \delta\in\{0, 1\}^D, \varphi\geq 0}\left(\frac{\gamma\left|c-w^T u\right|}{\varepsilon} + \frac{(1-\gamma)\varphi}{2700}\right)$$
        $$\theta_d u - \overline{\varphi}\leq(\theta_d e_I)\delta_d$$
        $$c-w^T u\leq\varepsilon$$
        $$w^T u-c\leq\varepsilon$$
        $$A^T u\geq e_M$$
        $$e_I^T u=k$$
        $$p\delta^T \leq 1-\alpha$$
    \end{frame}

    \begin{frame}{название слайда}
        текст\\
    \end{frame}
    
\end{document}